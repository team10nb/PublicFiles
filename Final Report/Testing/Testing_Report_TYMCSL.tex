% Options for packages loaded elsewhere
\PassOptionsToPackage{unicode}{hyperref}
\PassOptionsToPackage{hyphens}{url}
%
\documentclass[
]{article}
\usepackage{amsmath,amssymb}
\usepackage{lmodern}
\usepackage{iftex}
\ifPDFTeX
  \usepackage[T1]{fontenc}
  \usepackage[utf8]{inputenc}
  \usepackage{textcomp} % provide euro and other symbols
\else % if luatex or xetex
  \usepackage{unicode-math}
  \defaultfontfeatures{Scale=MatchLowercase}
  \defaultfontfeatures[\rmfamily]{Ligatures=TeX,Scale=1}
\fi
% Use upquote if available, for straight quotes in verbatim environments
\IfFileExists{upquote.sty}{\usepackage{upquote}}{}
\IfFileExists{microtype.sty}{% use microtype if available
  \usepackage[]{microtype}
  \UseMicrotypeSet[protrusion]{basicmath} % disable protrusion for tt fonts
}{}
\makeatletter
\@ifundefined{KOMAClassName}{% if non-KOMA class
  \IfFileExists{parskip.sty}{%
    \usepackage{parskip}
  }{% else
    \setlength{\parindent}{0pt}
    \setlength{\parskip}{6pt plus 2pt minus 1pt}}
}{% if KOMA class
  \KOMAoptions{parskip=half}}
\makeatother
\usepackage{xcolor}
\IfFileExists{xurl.sty}{\usepackage{xurl}}{} % add URL line breaks if available
\IfFileExists{bookmark.sty}{\usepackage{bookmark}}{\usepackage{hyperref}}
\hypersetup{
  hidelinks,
  pdfcreator={LaTeX via pandoc}}
\urlstyle{same} % disable monospaced font for URLs
\usepackage{longtable,booktabs,array}
\usepackage{calc} % for calculating minipage widths
% Correct order of tables after \paragraph or \subparagraph
\usepackage{etoolbox}
\makeatletter
\patchcmd\longtable{\par}{\if@noskipsec\mbox{}\fi\par}{}{}
\makeatother
% Allow footnotes in longtable head/foot
\IfFileExists{footnotehyper.sty}{\usepackage{footnotehyper}}{\usepackage{footnote}}
\makesavenoteenv{longtable}
\setlength{\emergencystretch}{3em} % prevent overfull lines
\providecommand{\tightlist}{%
  \setlength{\itemsep}{0pt}\setlength{\parskip}{0pt}}
\setcounter{secnumdepth}{-\maxdimen} % remove section numbering
\ifLuaTeX
  \usepackage{selnolig}  % disable illegal ligatures
\fi

\author{}
\date{}

\begin{document}

\tableofcontents

\hypertarget{header-n2}{%
\section{Testing Report}\label{header-n2}}

\hypertarget{header-n3}{%
\subsection{F.1 Introduction}\label{header-n3}}

The aim of testing is to program the software as we planned and
potentially prevent it from being broken. In general, testing is about
ensuring quality. Test-driven development (TDD) is an overall approach
that writes tests before implementing functionalities. This approach has
been considered as our primary methodology to ensure the quality of the
software. There are four major testing phases in this project, unit
testing, integration testing, release testing and acceptance testing.
Unit testing is responsible for individual pieces, ensuring the basic
functionality of each component. Integration testing checks whether
combinations of components work as expected. Release testing tests most
of the possible interactions to ensure the stability of the whole
system. Acceptance testing checks requirements and specifications by the
customer, indicating whether the software is accepted. In addition to
these testing phases, we apply continuous integration through the whole
process of development for spotting errors early and improving
efficiency. The following parts will introduce how these testing methods
were deployed during the development and problems we met with respective
remarks.

\hypertarget{header-n5}{%
\subsection{F.2 Unit Testing}\label{header-n5}}

As TDD instructs, developers in the team wrote unit tests for most basic
software components before coding any actual functionalities. Therefore,
unit tests work as the base of the whole project. By doing unit testing,
the team has a clearer view of what features a component is expected to
achieve. This can be illustrated by an example of a group in the team.
At the early stage, TDD was not taken seriously by some of the team
members. One group of two in the team did not follow the instruction of
TDD and wrote code directly without writing any unit test. The component
displays appropriately at first, but coding is painstaking as no clear
plan was made---the group of two modified their design multiple times.
After TDD was stressed to be vital, the group added unit tests for the
component but found a title in it was wrong. Compared to human eyes and
testing manually, automated unit testing helps design the code and
prevents potential mistakes by checking components each time they are
modified.

Specifically, unit testing in the project works for checking whether a
fundamental component contains expected texts, buttons and testing
whether functions inside a component run appropriately. Since we use
React as the JavaScript library, Jest is the project's primary unit
testing framework. React-testing-library is a testing utility that
encourages good testing practices and simplifies testing processes such
as rendering components and creating snapshots. It is possible to test a
combination of several components as well. As fundamental components are
already tested, mocking is utilised in testing combinations to avoid
repeated tests. Tested basic components and third-party components will
be mocked to avoid unnecessary rendering and unexpected errors.

All of the unit tests were firstly planned by documenting test plans in
detail. Any failed case and modification were also recorded in a test
log for future bug track convenience. Detailed test plans and logs can
be viewed in Appendix G.

Examples of unit testing points are as follows:

\begin{enumerate}
\def\labelenumi{\arabic{enumi}.}
\item
  Should contain specific text.
\item
  Should contain buttons.
\item
  A function should have been called after a button click event.
\item
  A subcomponent should have been called while rendering.
\item
  Functions should work as expected.
\end{enumerate}

\hypertarget{header-n21}{%
\subsubsection{F.2.1 Test Plan for Unit Testing}\label{header-n21}}

Please refer to Appendix G.

\hypertarget{header-n23}{%
\subsubsection{F.2.2 Test Log}\label{header-n23}}

Please refer to Appendix G.

\hypertarget{header-n26}{%
\subsection{F.3 Integration Testing}\label{header-n26}}

Integration testing tests subsystems. In this project, scenes and huge
combinations of multiple components are considered subsystems. Their
interfaces were tested by jest snapshot, and their interactions were
tested manually. Snapshot testing is a helpful tool to ensure a
subsystem has not been modified. If any of the elements were changed by
accident, the snapshot test would fail by comparing it to the old one.
Integration testing was often conducted at the end of a sprint and may
expose some bugs related to interaction. This is relatively helpful to
check whether a subsystem works as the specification expected.

Examples of Integration testing are as follows:

\begin{enumerate}
\def\labelenumi{\arabic{enumi}.}
\item
  Snapshot created and match with old one.
\item
  Test interactions manually in a subsystem to check them meet the
  requirement.
\end{enumerate}

\hypertarget{header-n34}{%
\subsubsection{F.3.1 Test Plan for Integration
Testing}\label{header-n34}}

Please refer to Appendix G.

\hypertarget{header-n36}{%
\subsubsection{F.3.2 Test Log for Integration
Testing}\label{header-n36}}

Please refer to Appendix G.

\hypertarget{header-n39}{%
\subsection{F.4 Release Testing}\label{header-n39}}

Release testing is expected to be conducted by an individual quality
assurance team that has not been involved in the system development.
However, due to the team's small size, all the team members have done
something related to the system. In this case, two members who focus
more on user interface would take responsibility for release testing.
They tested the software as a whole system manually to check whether the
system achieves all the specifications and works without abnormal.
Specifically, they took actions to simulate the user stories we defined.
Non-functional specifications were tested as well. Once it has been
done, the software is ready for acceptance testing and external use.

Three strategies taken are as follows:

\begin{enumerate}
\def\labelenumi{\arabic{enumi}.}
\item
  Performance driven.
\item
  Specification driven.
\item
  Scenario driven.
\end{enumerate}

\hypertarget{header-n50}{%
\subsubsection{F.4.1 Test Log}\label{header-n50}}

Test for I Can Sort

Release testing time: 2021/3/26

Part 1. First open

\begin{longtable}[]{@{}ll@{}}
\toprule
When first open this software, whether it will open a dialog window to
ask the user's level. & 1 \\
\midrule
\endhead
In the dialog window, click the left button, whether it will jump to
Tutorial module. & 1 \\
In the dialog window, click the right button, whether it will jump to
Procedure module. & 1 \\
\bottomrule
\end{longtable}

Part 2. Modules

\begin{longtable}[]{@{}ll@{}}
\toprule
In all modules, click the Settings button, whether a settings window
will pop up. & T \\
\midrule
\endhead
In all modules, whether the progress bars will show the progress of the
corresponding parts correctly. & T \\
In all modules, whether the recently accessed part will be stroked. &
T \\
In the Settings button, when click GITHUB ADDRESS: I-CAN-SORT button, it
will open a new browser window to the software's GitHub page. & T \\
In the Settings button, when click CONFIRM button, it will clear all
history. & T \\
In all modules, click the Help button, whether a help window will pop
up. & T \\
In Tutorial, click the Swap button, whether it goes to Swap page. & T \\
In Tutorial, click the Loop button, whether it goes to Loop page. & T \\
In Tutorial, click the Terminology button, whether it goes to
Terminology page. & T \\
In Procedure, click the Bubble sort button, whether it goes to Bubble
sort page. & T \\
In Procedure, click the Selection sort button, whether it goes to
Selection sort page. & T \\
In Procedure, click the Insertion sort button, whether it goes to
Insertion sort page. & T \\
In Procedure, click the Quick sort button, whether it goes to Quick sort
page. & T \\
In Procedure, click the Merge sort button, whether it goes to Merge sort
page. & T \\
In Procedure, click the Heap sort button, whether it goes to Heap sort
page. & T \\
In Correctness, click the Tutorial button, whether it goes to Tutorial
page. & T \\
In Correctness, click the Proof button, whether it goes to Proof page. &
T \\
\bottomrule
\end{longtable}

Part 3. Pages

\begin{longtable}[]{@{}ll@{}}
\toprule
In all pages, click the home button, whether it goes to the upper level
module correctly. & 1 \\
\midrule
\endhead
In Swap page, click the corresponding control buttons, whether the
animation will act as it is told to do. & 1 \\
In Swap page, when the animation is playing, whether the code is
highlighted correctly. & 1 \\
In Swap page, when the animation is playing, whether the values of
variables show correctly. & 1 \\
In Loop page, when first get into this page, whether a dialog window
pops up. & 1 \\
In Loop page, click the question marks, whether a dialog window pops up.
& 1 \\
In Loop page, click the corresponding control buttons, whether the
animation will act as it is told to do. & 1 \\
In Loop page, when the animation is playing, whether the code is
highlighted correctly. & 1 \\
In Terminology page, when first get into this page, whether a dialog
window pops up. & 1 \\
In Terminology page, when click left side catalogue buttons, whether it
goes to corresponding catalogue page correctly. & 1 \\
In Terminology page, when click back and next buttons, whether the pages
changes correctly. & 1 \\
In all subpages of Procedure, the Introduction part and the algorithm
shows correctly. & 1 \\
In all subpages of Procedure, the shuffle button of Operation and
Implementation part generates a random array correctly. & 1 \\
In all subpages of Procedure, the Export Quick Guide button exports a
PDF file correctly. & 1 \\
In Tutorial page of Correctness, when click left side catalogue buttons,
whether it goes to corresponding catalogue page correctly. & 1 \\
In all Tutorial pages of Correctness, when click back and next buttons,
whether the pages changes correctly. & 1 \\
In Input page of Tutorial, when click the shuffle button, whether it
will generate correct random inputs. & 1 \\
In Input page of Tutorial, when click the play button, whether it will
put correct random inputs into the algorithm. & 1 \\
In Example page of Tutorial, when click any algorithm, it will run
correctly. & 1 \\
\bottomrule
\end{longtable}

\hypertarget{header-n185}{%
\subsection{F.5 Acceptance Testing}\label{header-n185}}

Acceptance testing is the test after release testing and is done with
the clients. It aims to check whether the software meets stakeholders'
expectations and receive feedbacks from them. Team 10 prepared checklist
using similar strategy as release testing's and invited 6 stakeholders
from year 2 and year 4 and Dr Heshan to do the acceptance testing. Team
10 showed the software to them and asked them to try the software
freely. After trying the software, we discussed the problems with
stakeholders and took notes of their comments and feedbacks.

\hypertarget{header-n187}{%
\subsubsection{F.5.1 Feedback from Stakeholders}\label{header-n187}}

During the acceptance test on 27th March, the stakeholders gave Team 10
valuable suggestions. Some of them are listed as follows.

\begin{enumerate}
\def\labelenumi{\arabic{enumi}.}
\item
  In the Procedure module, most stakeholders think animation speed is
  too fast for stakeholders to understand, while some experienced
  stakeholders believe it is slow.
\item
  In the Implementation part of the Procedure module, some stakeholders
  find the quick guides (PDF files) helpful and would love to take a
  detailed look at them, while one user thinks it is unnecessary and
  complicated.
\item
  In the Correctness part, some stakeholders feel confused about what
  the software is doing. Most stakeholders with no algorithm correctness
  experience can not understand, especially the relationship among
  assertion, termination and correctness.
\item
  A user would like to use the keyboard to control the software. For
  example, press Enter to go to the next page or press Esc to close the
  pop-up window.
\item
  Most stakeholders do not prefer long sentences and paragraphs.
\item
  Most stakeholders would like more guides on how to use the software.
\item
  One thinks the user interface is plain and not so colourful.
\end{enumerate}

Team 10 conducted a acceptance survey for further justification. In
order to confirm with our supervisor, a acceptance checklist was also
agreed to be rubrics. Please refer to Survey Result in Appendix F.5.3 to
see the complete questionnaire. The acceptance checklist is provided in
Appendix H. Here take some representative data to illustrate.

In the result of the survey, no stakeholder finds the software very
difficult to use, but one says he/she may require a user guide. For the
modules' user experience, the Tutorial and Procedure modules are rated
4.5 out of 5, while the Correctness module is only rated 3.8. Team 10
analyses the reason for this to be, most of the stakeholders have no
experience in algorithm correctness. So Team 10 have to make it much
clearer and emphasise the definitions and relationships of the
professional terms.

Team 10 agrees on most of the suggestions and made improvement based on
them. After the improvements below were done, an acceptance testing
tested by Dr Heshan was passed. A corresponding acceptance checklist was
signed, in which our supervisor said she was satisfied and accept the
software to be released. The software was then released on the team's
GitHub page officially.

\hypertarget{header-n208}{%
\subsubsection{F.5.2 Improvement}\label{header-n208}}

\begin{enumerate}
\def\labelenumi{\arabic{enumi}.}
\item
  \textbf{First In page:} Extra information in explaining our software's
  purpose is added.
\item
  \textbf{Tutorial Swap:} The Swap page's introduction has been
  modified. Now it is clearly instructed to press the play button to
  learn swap.
\item
  \textbf{Procedure Operation:} Pressing Enter to input an array is now
  realizable. Now users can use both comma and white space to split two
  numbers when inputting.
\item
  \textbf{Procedure Implementation:} An "in increasing order" option is
  added to clarify the sorting order.
\item
  \textbf{Procedure Implementation:} Merge and Heap's pseudocode
  comments are added, informing users of checking the detail of merge
  and maxHeapify in quick guides.
\item
  \textbf{Animation:} The speed of animation is carefully adjusted to
  satisfy inexperienced users.
\item
  \textbf{Correctness Termination:} The animation for the terminating
  example is now changed. Now the two algorithms are both swap
  algorithms.
\item
  \textbf{More pop-up windows:} More pop-up windows are added in the
  First In page to inform the users how to utilize modules.
\item
  \textbf{Correctness Proof:} Extra information about the link between
  proof and assertions are added. It now stresses that this is an aid
  tool that is not theoretical enough.
\item
  \textbf{Selection sort:} One non-clear assertion is rewritten.
\item
  \textbf{Quicksort:} Quicksort is now presented recursively, being made
  consistent with merge and heap.
\end{enumerate}

\hypertarget{header-n233}{%
\subsubsection{F.5.3 Survey Result}\label{header-n233}}

\begin{longtable}[]{@{}lll@{}}
\toprule
\textbf{Survey statistics (1 for disagree, 5 for agree)} & & \\
\midrule
\endhead
\textbf{1. I found the software unnecessarily complex } & & \\
\textbf{Options} & Percentage\% & Subtotal \\
\textbf{1} & 42.9\% & 3 \\
\textbf{2} & 57.1\% & 4 \\
\textbf{3} & 0.0\% & 0 \\
\textbf{4} & 0.0\% & 0 \\
\textbf{5} & 0.0\% & 0 \\
\textbf{Effective amount} & & 7 \\
& & \\
\textbf{2. I think the software is easy to use } & & \\
\textbf{Options} & Percentage\% & Subtotal \\
\textbf{1} & 0.0\% & 0 \\
\textbf{2} & 0.0\% & 0 \\
\textbf{3} & 14.3\% & 1 \\
\textbf{4} & 42.9\% & 3 \\
\textbf{5} & 42.9\% & 3 \\
\textbf{Effective amount} & & 7 \\
& & \\
\textbf{3. I need a user manual to use this software} & & \\
\textbf{Options} & Percentage\% & Subtotal \\
\textbf{1} & 28.6\% & 2 \\
\textbf{2} & 28.6\% & 2 \\
\textbf{3} & 0.0\% & 0 \\
\textbf{4} & 42.9\% & 3 \\
\textbf{5} & 0.0\% & 0 \\
\textbf{Effective amount} & & 7 \\
& & \\
\textbf{4. I understand the meanings of buttons} & & \\
\textbf{Options} & Percentage\% & Subtotal \\
\textbf{1} & 0.0\% & 0 \\
\textbf{2} & 0.0\% & 0 \\
\textbf{3} & 28.6\% & 2 \\
\textbf{4} & 14.3\% & 1 \\
\textbf{5} & 57.1\% & 4 \\
\textbf{Effective amount} & & 7 \\
& & \\
\textbf{5. I think the UI is consistent in this software} & & \\
\textbf{Options} & Percentage\% & Subtotal \\
\textbf{1} & 0.0\% & 0 \\
\textbf{2} & 0.0\% & 0 \\
\textbf{3} & 0.0\% & 0 \\
\textbf{4} & 14.3\% & 1 \\
\textbf{5} & 85.7\% & 6 \\
\textbf{Effective amount} & & 7 \\
& & \\
\textbf{6. I found the various functions in this software were well
integrated } & & \\
\textbf{Options} & Percentage\% & Subtotal \\
\textbf{1} & 0.0\% & 0 \\
\textbf{2} & 0.0\% & 0 \\
\textbf{3} & 0.0\% & 0 \\
\textbf{4} & 42.9\% & 3 \\
\textbf{5} & 57.1\% & 4 \\
\textbf{Effective amount} & & 7 \\
& & \\
\textbf{7. I think that I would like to use this software frequently } &
& \\
\textbf{Options} & Percentage\% & Subtotal \\
\textbf{1} & 0.0\% & 0 \\
\textbf{2} & 14.3\% & 1 \\
\textbf{3} & 28.6\% & 2 \\
\textbf{4} & 42.9\% & 3 \\
\textbf{5} & 14.3\% & 1 \\
\textbf{Effective amount} & & 7 \\
& & \\
\textbf{8. I think the Tutorial module is useful } & & \\
\textbf{Options} & Percentage\% & Subtotal \\
\textbf{1} & 0.0\% & 0 \\
\textbf{2} & 0.0\% & 0 \\
\textbf{3} & 14.3\% & 1 \\
\textbf{4} & 42.9\% & 3 \\
\textbf{5} & 42.9\% & 3 \\
\textbf{Effective amount} & & 7 \\
& & \\
\textbf{9. I think the Process module is clear} & & \\
\textbf{Options} & Percentage\% & Subtotal \\
\textbf{1} & 0.0\% & 0 \\
\textbf{2} & 0.0\% & 0 \\
\textbf{3} & 14.3\% & 1 \\
\textbf{4} & 14.3\% & 1 \\
\textbf{5} & 71.4\% & 5 \\
\textbf{Effective amount} & & 7 \\
& & \\
\textbf{10. I think the Correctness module is useful} & & \\
\textbf{Options} & Percentage\% & Subtotal \\
\textbf{1} & 0.0\% & 0 \\
\textbf{2} & 0.0\% & 0 \\
\textbf{3} & 28.6\% & 2 \\
\textbf{4} & 57.1\% & 4 \\
\textbf{5} & 14.3\% & 1 \\
\textbf{Effective amount} & & 7 \\
& & \\
\textbf{11. I really learned something about sorting algorithms by using
this software } & & \\
\textbf{Options} & Percentage\% & Subtotal \\
\textbf{1} & 0.0\% & 0 \\
\textbf{2} & 0.0\% & 0 \\
\textbf{3} & 0.0\% & 0 \\
\textbf{4} & 28.6\% & 2 \\
\textbf{5} & 71.4\% & 5 \\
\textbf{Effective amount} & & 7 \\
& & \\
\textbf{12. I encountered confusing parts when using this software} &
& \\
\textbf{Options} & Percentage\% & Subtotal \\
\textbf{1} & 14.3\% & 1 \\
\textbf{2} & 28.6\% & 2 \\
\textbf{3} & 14.3\% & 1 \\
\textbf{4} & 42.9\% & 3 \\
\textbf{5} & 0.0\% & 0 \\
\textbf{Effective amount} & & 7 \\
& & \\
\textbf{13. I will recommend this software to my friends} & & \\
\textbf{Options} & Percentage\% & Subtotal \\
\textbf{1} & 0.0\% & 0 \\
\textbf{2} & 0.0\% & 0 \\
\textbf{3} & 0.0\% & 0 \\
\textbf{4} & 57.1\% & 4 \\
\textbf{5} & 42.9\% & 3 \\
\textbf{Effective amount} & & 7 \\
& & \\
\textbf{14. Any suggestions } & & \\
\textbf{Number} & Answer & Time \\
\textbf{6} & All Good. Please keep developing! &
2021-03-28T00:39:37+08:00 \\
\textbf{4} & no so far & 2021-03-27T15:47:11+08:00 \\
\textbf{3} & no & 2021-03-27T15:45:03+08:00 \\
\textbf{2} & no & 2021-03-27T15:44:41+08:00 \\
\bottomrule
\end{longtable}

\end{document}
